Tropical cyclones, especially fully developed hurricanes,
exhibit quasi-stationary features with respect to the position of the storm center.
Such features include the eye-wall cloud, precipitation bands, and wind speed maxima.
Forecast model analysis and verification can thus benefit from statistics
aggregated on a moving range-azimuth grid centered on points along the storm track.
In this poster we will describe a new TC moving grid analysis tool
developed within the Developmental Testbed Center (DTC)
Model Evaluation Tools (MET) software suite.
MET is highly configurable and provides a variety of verification techniques
including standard verification scores comparing gridded model data
to gridded and point observations or re-analyses.
MET was chosen as the core of the National Oceanic and Atmospheric Administration (NOAA)
Unified Forecast System (UFS).
The larger system, METplus, includes the core MET tools,
python wrappers for automation over verification tasks,
a results database (METdb),
aggregation tools (METcalcpy),
and visualization tools (METplotpy, METviewer, METexpress).
\newline

This MET-TC tool supports several re-gridding methods,
storm track selection, and data aggregation or stratification based
on user specified criteria such as forecast initialization time.
Derived fields such as azimuthal means and the tangential
and radial wind components are computed.
The moving grid may also be rescaled along the track.
For example, the radial grid spacing may be set
as a factor of the radius of maximum winds.
\newline

As a case study we will present a 12-day (Sep. 28 – Oct. 9, 2016) analysis
of the Finite-Volume Cubed-Sphere 3 Global Forecast System (FV3GFS) forecasts
for Hurricane Matthew,
which intensified to a category 5 hurricane in the Caribbean Sea
and tracked along the southeast coast of the US,
where it continued to produce heavy rainfall and widespread flooding.
This dataset was created as a baseline case
for the Model Evaluation for Research Innovation Transition (MERIT) project,
with storm tracks generated from the Geophysical Fluid Dynamics Laboratory (GFDL)
vortex tracker.
The METplus modules, METcalcpy and METplotpy,
are used to examine the evolution of various surface
and pressure level quantities such as sea-level pressure,
precipitation, and surface and upper air wind fields.
Radial-vertical cross sections of azimuthally averaged quantities are also computed;
in particular, example statistics
and visualizations are generated for tangential winds,
radial and vertical mass and water vapor fluxes,
and equivalent potential temperature.
